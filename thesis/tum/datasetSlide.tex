\documentclass{beamer}
\usepackage{hyperref}

\begin{document}

    \title{MRI Breast Segmentation Dataset Comparison}
    \author{Bikem Çamli}
    \date{\today}


    \frame{\titlepage}
    \begin{frame}
        \frametitle{ \href{https://nbia.cancerimagingarchive.net/viewer/?study=1.3.6.1.4.1.14519.5.2.1.295822356709841808204887570788704838212&series=1.3.6.1.4.1.14519.5.2.1.210143579418064304537763273309118865786&token=d3ebb6d2-30da-48d5-a295-0a4fa82d310a}{\underline{\textcolor{blue}{Duke-Breast-Cancer-MRI}}}}

        \begin{block}{Dataset Overview}
            \begin{itemize}
                \item \textbf{Collection:} Single-institutional, retrospective collection of 922 biopsy-confirmed breast cancer patients, for a decade
                \item \textbf{Quality:} Annotations provided by 8 radiologists, further reviewed and modified for accuracy.
                \item \textbf{Number of Images:} 773,888 DICOM images.
            \end{itemize}
        \end{block}

        \begin{block}{Dataset Components}
            \begin{itemize}
                \item Demographic, clinical, pathology, treatment, outcomes, and genomic data.
                \item Pre-operative dynamic contrast enhanced (DCE)-MRI: T1-weighted and post-contrast sequences.
                \item Locations of lesions in DCE-MRI: Annotations by radiologists.
                \item Imaging features from DCE-MRI: 529 computer-extracted features: size, shape, texture, and enhancement of both the tumor and the surrounding tissue.
            \end{itemize}
        \end{block}
    \end{frame}

    \begin{frame}
        \frametitle{ \href{https://nbia.cancerimagingarchive.net/viewer/?study=1.3.6.1.4.1.14519.5.2.1.186051521067863971269584893740842397538&series=1.3.6.1.4.1.14519.5.2.1.185777849803665244536713316058283493877&token=5a3745db-7523-4fb2-bb86-c9bd67585b61}{\underline{\textcolor{blue}{Duke-Breast-Cancer-MRI}}}}

        \begin{block}{Image Annotations}
            \begin{itemize}
                \item Drawn by 8 radiologists using a GUI developed in MATLAB.
                \item 3D boxes around areas of mass and non-mass enhancement.
                \item The MRI sequences that were involved in annotation were: (a) pre-contrast, (b) first post-contrast, and (c) subtracted
            \end{itemize}
            \item Two annotation phases:
            \begin{itemize}
                \item For 271 patients, 6 radiologists annotated
                    \begin{itemize}
                        \item Up to five lesions were annotated per patient.
                        \item Biopsied tumor selection involved reviewing radiology and pathology reports.
                        \item Largest tumor selected in  multiple biopsies
                    \end{itemize}
                \item For 651 patients 4 radiologists annotated
                    \begin{itemize}
                        \item A modified annotation procedure was implemented
                        \item Radiologists were given the locations of the biopsies, instructed to annotate the largest biopsied lesion.
                    \end{itemize}
            \end{itemize}
        \end{block}
    \end{frame}

    \begin{frame}
        \frametitle{   \href{https://nbia.cancerimagingarchive.net/viewer/?study=1.3.6.1.4.1.14519.5.2.1.7695.4164.273481108061907436142425976120&series=1.3.6.1.4.1.14519.5.2.1.7695.4164.327894017749939121381164595822&token=c2744136-6276-4bb6-8111-2e85dcbda6cf}{\underline{\textcolor{blue}{ACRIN 6698/I-SPY2}}}}
        \begin{block}{Dataset Overview}
            \begin{itemize}
                \item \textbf{Collection:} From 10 instution collection of a 406 women with invasive breast cancer, for 3 years
                \item \textbf{Quality:} Image quality control system consists of three sequential but independent assessment stages: protocol compliance, image quality and usability, and ROI confidence
                \item \textbf{Number of Images:} 2,911,334 images.
            \end{itemize}
        \end{block}

        \begin{block}{Dataset Components}
            \begin{itemize}
                \item Patient demographic, clinical and outcome data files for limited set of patients
                \item T2-weighted imaging, diffusion-weighted imaging (DWI), and dynamic contrast-enhanced (DCE)
                \item Manual DWI Whole-Tumor Segmentation
                \item Test-Retest Data to allow evaluation of repeatability and reproducibility of new DWI metrics and analysis techniques
            \end{itemize}
        \end{block}
    \end{frame}

    \begin{frame}
        \frametitle{   \href{https://nbia.cancerimagingarchive.net/viewer/?study=1.3.6.1.4.1.14519.5.2.1.7695.4164.273481108061907436142425976120&series=1.3.6.1.4.1.14519.5.2.1.7695.4164.118116703802628474808498368280&token=c2744136-6276-4bb6-8111-2e85dcbda6cf}{\underline{\textcolor{blue}{ACRIN 6698/I-SPY2}}}}
        \begin{block}{Image Annotations}
            \begin{itemize}
                \item Manually delimited tumor segmentations from the primary study analysis (for all studies rated as analyzable in the QC evaluation)
                \item Region definition was done at the UCSF processing lab using in-house software tools
                \item ROIs defined for published primary analysis
                \item Tumor was identified on post-contrast DCE subtraction images and then localized on the ADC map
                \item Multi-slice, whole-tumor regions of interest (ROIs) were manually defined by selecting regions with low ADC and hyperintensity on a high b-value DWI
                \item The segmentations are provided both as DICOM SEG objects and as DICOM MRI objects on TCIA
            \end{itemize}
        \end{block}
    \end{frame}


    \begin{frame}
        \frametitle{ \href{https://nbia.cancerimagingarchive.net/viewer/?study=1.3.6.1.4.1.14519.5.2.1.7695.4164.273481108061907436142425976120&series=1.3.6.1.4.1.14519.5.2.1.7695.4164.327894017749939121381164595822&token=c2744136-6276-4bb6-8111-2e85dcbda6cf}{\underline{\textcolor{blue}{I-SPY 2 Breast Cancer Trial}}}}

        \begin{block}{Dataset Overview}
            \begin{itemize}
                \item \textbf{Collection:} From 22 health center collection of a 719 patients with invasive breast cancer, along 2010-2016
                \item \textbf{Quality:} Over 95 \% of the DCE imaging data met acceptance criteria for analysis of functional tumor volume (FTV)
                \item \textbf{Number of Images:} 5,586,493 images.
            \end{itemize}
        \end{block}
        \begin{block}{Dataset Components}
            \begin{itemize}
                \item T2-weighted imaging, diffusion-weighted imaging (DWI), and dynamic contrast-enhanced (DCE)
                \item Early-treatment (T1, optional test/retest visit), Mid-treatment (T2), Post-treatment (T3)
                \item Derived objects from the DCE acquisitions, enhancement maps and functional tumor volume (FTV) analysis mask
            \end{itemize}
        \end{block}
    \end{frame}

    \begin{frame}
        \frametitle{ \href{https://nbia.cancerimagingarchive.net/viewer/?study=1.3.6.1.4.1.14519.5.2.1.296160879836167288241026943725310014603&series=1.3.6.1.4.1.14519.5.2.1.146922122261172320578401952552745618812&token=3ca5b192-1ca8-45cc-95c8-84221a74936f}{\underline{\textcolor{blue}{I-SPY 2 Breast Cancer Trial}}}}
        \begin{block}{Image Annotations}
            \begin{itemize}
                \item FTV Analysis Masks:
                    \begin{itemize}
                        \item Bit-encoded segmentations
                        \item Encode masking steps in primary FTV analysis
                        \item DICOM SEG objects in a separate series
                    \end{itemize}
                \item Segmentation Details:
                    \begin{itemize}
                        \item Pre-contrast background thresholding
                        \item Minimum percent enhancement thresholding
                        \item Manually defined rectangular volume of interest (VOI) for enhancing tumor analysis
                        \item Manually defined "OMIT" regions to exclude non-tumor enhancing regions
                    \end{itemize}
            \end{itemize}
        \end{block}
    \end{frame}


    \begin{frame}
        \frametitle{Dataset Comparison}
        \begin{table}[h]
            \centering
            \resizebox{\textwidth}{!}{
                \begin{tabular}{|p{3cm}|p{2.5cm}|p{2.5cm}|p{2.5cm}|}
                    \hline
                    Dataset            & \href{https://wiki.cancerimagingarchive.net/pages/viewpage.action?pageId=70226903#702269033366bb5668004bc4a9f98c427c168215}{\underline{\textcolor{blue}{Duke}}} & \href{https://wiki.cancerimagingarchive.net/pages/viewpage.action?pageId=50135447#50135447a3ebdd1155ef41b3b0650f8d3cd77675}{\underline{\textcolor{blue}{ACRIN 6698}}} & \href{https://wiki.cancerimagingarchive.net/pages/viewpage.action?pageId=70230072#702300722a08eb787052462dab82d0330507e048}{\underline{\textcolor{blue}{ISPY2}}} \\
                    \hline
                    Modalities         & MR, SEG                                                                                                                                                         & MR, SEG                                                                                                                                                               & MR, SEG                                                                                                                                                          \\
                    \hline
                    \# of Participants & 922                                                                                                                                                             & 385                                                                                                                                                                   & 719                                                                                                                                                              \\
                    \hline
                    \# of Studies      & 922                                                                                                                                                             & 1,123                                                                                                                                                                 & 2,688                                                                                                                                                            \\
                    \hline
                    \# of Series       & 5,161                                                                                                                                                           & 18,747                                                                                                                                                                & 32,411                                                                                                                                                           \\
                    \hline
                    Images             & 773,888                                                                                                                                                         & 2,911,334                                                                                                                                                             & 5,586,493                                                                                                                                                        \\
                    \hline
                    Image Size         & 368.4 GB                                                                                                                                                        & 842 GB                                                                                                                                                                & 1.6 TB                                                                                                                                                           \\
                    \hline
                \end{tabular}
            }
        \end{table}
    \end{frame}

\end{document}